% Options for packages loaded elsewhere
\PassOptionsToPackage{unicode}{hyperref}
\PassOptionsToPackage{hyphens}{url}
%
\documentclass[
]{article}
\usepackage{amsmath,amssymb}
\usepackage{iftex}
\ifPDFTeX
  \usepackage[T1]{fontenc}
  \usepackage[utf8]{inputenc}
  \usepackage{textcomp} % provide euro and other symbols
\else % if luatex or xetex
  \usepackage{unicode-math} % this also loads fontspec
  \defaultfontfeatures{Scale=MatchLowercase}
  \defaultfontfeatures[\rmfamily]{Ligatures=TeX,Scale=1}
\fi
\usepackage{lmodern}
\ifPDFTeX\else
  % xetex/luatex font selection
\fi
% Use upquote if available, for straight quotes in verbatim environments
\IfFileExists{upquote.sty}{\usepackage{upquote}}{}
\IfFileExists{microtype.sty}{% use microtype if available
  \usepackage[]{microtype}
  \UseMicrotypeSet[protrusion]{basicmath} % disable protrusion for tt fonts
}{}
\makeatletter
\@ifundefined{KOMAClassName}{% if non-KOMA class
  \IfFileExists{parskip.sty}{%
    \usepackage{parskip}
  }{% else
    \setlength{\parindent}{0pt}
    \setlength{\parskip}{6pt plus 2pt minus 1pt}}
}{% if KOMA class
  \KOMAoptions{parskip=half}}
\makeatother
\usepackage{xcolor}
\usepackage[margin=1in]{geometry}
\usepackage{color}
\usepackage{fancyvrb}
\newcommand{\VerbBar}{|}
\newcommand{\VERB}{\Verb[commandchars=\\\{\}]}
\DefineVerbatimEnvironment{Highlighting}{Verbatim}{commandchars=\\\{\}}
% Add ',fontsize=\small' for more characters per line
\usepackage{framed}
\definecolor{shadecolor}{RGB}{248,248,248}
\newenvironment{Shaded}{\begin{snugshade}}{\end{snugshade}}
\newcommand{\AlertTok}[1]{\textcolor[rgb]{0.94,0.16,0.16}{#1}}
\newcommand{\AnnotationTok}[1]{\textcolor[rgb]{0.56,0.35,0.01}{\textbf{\textit{#1}}}}
\newcommand{\AttributeTok}[1]{\textcolor[rgb]{0.13,0.29,0.53}{#1}}
\newcommand{\BaseNTok}[1]{\textcolor[rgb]{0.00,0.00,0.81}{#1}}
\newcommand{\BuiltInTok}[1]{#1}
\newcommand{\CharTok}[1]{\textcolor[rgb]{0.31,0.60,0.02}{#1}}
\newcommand{\CommentTok}[1]{\textcolor[rgb]{0.56,0.35,0.01}{\textit{#1}}}
\newcommand{\CommentVarTok}[1]{\textcolor[rgb]{0.56,0.35,0.01}{\textbf{\textit{#1}}}}
\newcommand{\ConstantTok}[1]{\textcolor[rgb]{0.56,0.35,0.01}{#1}}
\newcommand{\ControlFlowTok}[1]{\textcolor[rgb]{0.13,0.29,0.53}{\textbf{#1}}}
\newcommand{\DataTypeTok}[1]{\textcolor[rgb]{0.13,0.29,0.53}{#1}}
\newcommand{\DecValTok}[1]{\textcolor[rgb]{0.00,0.00,0.81}{#1}}
\newcommand{\DocumentationTok}[1]{\textcolor[rgb]{0.56,0.35,0.01}{\textbf{\textit{#1}}}}
\newcommand{\ErrorTok}[1]{\textcolor[rgb]{0.64,0.00,0.00}{\textbf{#1}}}
\newcommand{\ExtensionTok}[1]{#1}
\newcommand{\FloatTok}[1]{\textcolor[rgb]{0.00,0.00,0.81}{#1}}
\newcommand{\FunctionTok}[1]{\textcolor[rgb]{0.13,0.29,0.53}{\textbf{#1}}}
\newcommand{\ImportTok}[1]{#1}
\newcommand{\InformationTok}[1]{\textcolor[rgb]{0.56,0.35,0.01}{\textbf{\textit{#1}}}}
\newcommand{\KeywordTok}[1]{\textcolor[rgb]{0.13,0.29,0.53}{\textbf{#1}}}
\newcommand{\NormalTok}[1]{#1}
\newcommand{\OperatorTok}[1]{\textcolor[rgb]{0.81,0.36,0.00}{\textbf{#1}}}
\newcommand{\OtherTok}[1]{\textcolor[rgb]{0.56,0.35,0.01}{#1}}
\newcommand{\PreprocessorTok}[1]{\textcolor[rgb]{0.56,0.35,0.01}{\textit{#1}}}
\newcommand{\RegionMarkerTok}[1]{#1}
\newcommand{\SpecialCharTok}[1]{\textcolor[rgb]{0.81,0.36,0.00}{\textbf{#1}}}
\newcommand{\SpecialStringTok}[1]{\textcolor[rgb]{0.31,0.60,0.02}{#1}}
\newcommand{\StringTok}[1]{\textcolor[rgb]{0.31,0.60,0.02}{#1}}
\newcommand{\VariableTok}[1]{\textcolor[rgb]{0.00,0.00,0.00}{#1}}
\newcommand{\VerbatimStringTok}[1]{\textcolor[rgb]{0.31,0.60,0.02}{#1}}
\newcommand{\WarningTok}[1]{\textcolor[rgb]{0.56,0.35,0.01}{\textbf{\textit{#1}}}}
\usepackage{graphicx}
\makeatletter
\def\maxwidth{\ifdim\Gin@nat@width>\linewidth\linewidth\else\Gin@nat@width\fi}
\def\maxheight{\ifdim\Gin@nat@height>\textheight\textheight\else\Gin@nat@height\fi}
\makeatother
% Scale images if necessary, so that they will not overflow the page
% margins by default, and it is still possible to overwrite the defaults
% using explicit options in \includegraphics[width, height, ...]{}
\setkeys{Gin}{width=\maxwidth,height=\maxheight,keepaspectratio}
% Set default figure placement to htbp
\makeatletter
\def\fps@figure{htbp}
\makeatother
\setlength{\emergencystretch}{3em} % prevent overfull lines
\providecommand{\tightlist}{%
  \setlength{\itemsep}{0pt}\setlength{\parskip}{0pt}}
\setcounter{secnumdepth}{-\maxdimen} % remove section numbering
\ifLuaTeX
  \usepackage{selnolig}  % disable illegal ligatures
\fi
\IfFileExists{bookmark.sty}{\usepackage{bookmark}}{\usepackage{hyperref}}
\IfFileExists{xurl.sty}{\usepackage{xurl}}{} % add URL line breaks if available
\urlstyle{same}
\hypersetup{
  pdftitle={Predict},
  pdfauthor={Mansour haghi},
  hidelinks,
  pdfcreator={LaTeX via pandoc}}

\title{Predict}
\author{Mansour haghi}
\date{2023-08-16}

\begin{document}
\maketitle

\hypertarget{initial-configuration}{%
\subsection{Initial configuration}\label{initial-configuration}}

In the first step, the required information has been downloaded and some
corrections have been made in the data.

\#1-) Packages required

\begin{Shaded}
\begin{Highlighting}[]
\NormalTok{IscaretInstalled }\OtherTok{\textless{}{-}} \FunctionTok{require}\NormalTok{(}\StringTok{"caret"}\NormalTok{)}
\end{Highlighting}
\end{Shaded}

\begin{verbatim}
## Loading required package: caret
\end{verbatim}

\begin{verbatim}
## Loading required package: ggplot2
\end{verbatim}

\begin{verbatim}
## Loading required package: lattice
\end{verbatim}

\begin{Shaded}
\begin{Highlighting}[]
\ControlFlowTok{if}\NormalTok{(}\SpecialCharTok{!}\NormalTok{IscaretInstalled)\{}
  \FunctionTok{install.packages}\NormalTok{(}\StringTok{"caret"}\NormalTok{)}
  \FunctionTok{library}\NormalTok{(}\StringTok{"caret"}\NormalTok{)}
\NormalTok{\}}
\NormalTok{IsrandomForestInstalled }\OtherTok{\textless{}{-}} \FunctionTok{require}\NormalTok{(}\StringTok{"randomForest"}\NormalTok{)}
\end{Highlighting}
\end{Shaded}

\begin{verbatim}
## Loading required package: randomForest
\end{verbatim}

\begin{verbatim}
## randomForest 4.7-1.1
\end{verbatim}

\begin{verbatim}
## Type rfNews() to see new features/changes/bug fixes.
\end{verbatim}

\begin{verbatim}
## 
## Attaching package: 'randomForest'
\end{verbatim}

\begin{verbatim}
## The following object is masked from 'package:ggplot2':
## 
##     margin
\end{verbatim}

\begin{Shaded}
\begin{Highlighting}[]
\ControlFlowTok{if}\NormalTok{(}\SpecialCharTok{!}\NormalTok{IsrandomForestInstalled)\{}
  \FunctionTok{install.packages}\NormalTok{(}\StringTok{"randomForest"}\NormalTok{)}
  \FunctionTok{library}\NormalTok{(}\StringTok{"randomForest"}\NormalTok{)}
\NormalTok{\}}
\NormalTok{IsRpartInstalled }\OtherTok{\textless{}{-}} \FunctionTok{require}\NormalTok{(}\StringTok{"rpart"}\NormalTok{)        }
\end{Highlighting}
\end{Shaded}

\begin{verbatim}
## Loading required package: rpart
\end{verbatim}

\begin{Shaded}
\begin{Highlighting}[]
\ControlFlowTok{if}\NormalTok{(}\SpecialCharTok{!}\NormalTok{IsRpartInstalled)\{}
  \FunctionTok{install.packages}\NormalTok{(}\StringTok{"rpart"}\NormalTok{)}
  \FunctionTok{library}\NormalTok{(}\StringTok{"rpart"}\NormalTok{)}
\NormalTok{\}}
\NormalTok{IsRpartPlotInstalled }\OtherTok{\textless{}{-}} \FunctionTok{require}\NormalTok{(}\StringTok{"rpart.plot"}\NormalTok{)}
\end{Highlighting}
\end{Shaded}

\begin{verbatim}
## Loading required package: rpart.plot
\end{verbatim}

\begin{Shaded}
\begin{Highlighting}[]
\ControlFlowTok{if}\NormalTok{(}\SpecialCharTok{!}\NormalTok{IsRpartPlotInstalled)\{}
  \FunctionTok{install.packages}\NormalTok{(}\StringTok{"rpart.plot"}\NormalTok{)}
  \FunctionTok{library}\NormalTok{(}\StringTok{"rpart.plot"}\NormalTok{)}
\NormalTok{\}}

\CommentTok{\#Set seed for reproducability}
\FunctionTok{set.seed}\NormalTok{(}\DecValTok{1000}\NormalTok{)}
\end{Highlighting}
\end{Shaded}

\#2-) Data Processing

\#Load the data and Data variables

\begin{Shaded}
\begin{Highlighting}[]
\NormalTok{ trainUrl }\OtherTok{\textless{}{-}} \StringTok{"http://d396qusza40orc.cloudfront.net/predmachlearn/pml{-}training.csv"}  
\NormalTok{testUrl }\OtherTok{\textless{}{-}} \StringTok{"http://d396qusza40orc.cloudfront.net/predmachlearn/pml{-}testing.csv"}
\end{Highlighting}
\end{Shaded}

\hypertarget{download-data}{%
\section{Download data}\label{download-data}}

\begin{Shaded}
\begin{Highlighting}[]
\NormalTok{training }\OtherTok{\textless{}{-}} \FunctionTok{read.csv}\NormalTok{(}\FunctionTok{url}\NormalTok{(trainUrl))}
\NormalTok{testing }\OtherTok{\textless{}{-}} \FunctionTok{read.csv}\NormalTok{(}\FunctionTok{url}\NormalTok{(testUrl))}
\end{Highlighting}
\end{Shaded}

\#3-)Clean data

\begin{Shaded}
\begin{Highlighting}[]
\CommentTok{\#Delete missing values and Remove variables(columns) with zero number}
\NormalTok{training1 }\OtherTok{\textless{}{-}} \FunctionTok{read.csv}\NormalTok{(}\FunctionTok{url}\NormalTok{(trainUrl), }\AttributeTok{na.strings=}\FunctionTok{c}\NormalTok{(}\StringTok{"NA"}\NormalTok{,}\StringTok{"\#DIV/0!"}\NormalTok{,}\StringTok{""}\NormalTok{))  }
\NormalTok{testing1 }\OtherTok{\textless{}{-}} \FunctionTok{read.csv}\NormalTok{(}\FunctionTok{url}\NormalTok{(testUrl), }\AttributeTok{na.strings=}\FunctionTok{c}\NormalTok{(}\StringTok{"NA"}\NormalTok{,}\StringTok{"\#DIV/0!"}\NormalTok{,}\StringTok{""}\NormalTok{))}
\NormalTok{training2}\OtherTok{\textless{}{-}}\NormalTok{training1[,}\FunctionTok{colSums}\NormalTok{(}\FunctionTok{is.na}\NormalTok{(training1)) }\SpecialCharTok{==} \DecValTok{0}\NormalTok{]}
\NormalTok{testing2 }\OtherTok{\textless{}{-}}\NormalTok{testing1[,}\FunctionTok{colSums}\NormalTok{(}\FunctionTok{is.na}\NormalTok{(testing1)) }\SpecialCharTok{==} \DecValTok{0}\NormalTok{]}

\CommentTok{\#Delete seven first columns that are not predictors}
\NormalTok{training3   }\OtherTok{\textless{}{-}}\NormalTok{training2[,}\SpecialCharTok{{-}}\FunctionTok{c}\NormalTok{(}\DecValTok{1}\SpecialCharTok{:}\DecValTok{7}\NormalTok{)]}
\NormalTok{testing3 }\OtherTok{\textless{}{-}}\NormalTok{testing2[,}\SpecialCharTok{{-}}\FunctionTok{c}\NormalTok{(}\DecValTok{1}\SpecialCharTok{:}\DecValTok{7}\NormalTok{)]}
\end{Highlighting}
\end{Shaded}

\#4-) Cross-validation In this section split the training data in
training (75\%) and testing (25\%) data) subsets.

\begin{Shaded}
\begin{Highlighting}[]
\NormalTok{SplitTrain }\OtherTok{\textless{}{-}} \FunctionTok{createDataPartition}\NormalTok{(}\AttributeTok{y=}\NormalTok{training3}\SpecialCharTok{$}\NormalTok{classe, }\AttributeTok{p=}\FloatTok{0.75}\NormalTok{, }\AttributeTok{list=}\ConstantTok{FALSE}\NormalTok{)    }
\NormalTok{STraining }\OtherTok{\textless{}{-}}\NormalTok{ training3[SplitTrain, ]}
\NormalTok{STesting }\OtherTok{\textless{}{-}}\NormalTok{ training3[}\SpecialCharTok{{-}}\NormalTok{SplitTrain, ]}
\FunctionTok{dim}\NormalTok{(STraining)}
\FunctionTok{dim}\NormalTok{(STesting)}
\end{Highlighting}
\end{Shaded}

\hypertarget{exploratory-analysis}{%
\section{Exploratory analysis}\label{exploratory-analysis}}

The variable \texttt{classe} contains 5 levels. The plot of the outcome
variable shows the frequency of each levels in the subTraining data.

\begin{Shaded}
\begin{Highlighting}[]
\FunctionTok{plot}\NormalTok{(}\FunctionTok{as.factor}\NormalTok{(STraining}\SpecialCharTok{$}\NormalTok{classe), }\AttributeTok{col=}\StringTok{"green"}\NormalTok{, }\AttributeTok{xlab=}\StringTok{"classe\_levels"}\NormalTok{, }\AttributeTok{ylab=}\StringTok{"Frequency"}\NormalTok{,}\AttributeTok{main=}\StringTok{"Levels\_classe"}\NormalTok{)}
\end{Highlighting}
\end{Shaded}

\includegraphics{perdict_files/figure-latex/Analysis-1.pdf}

\#5-) Prediction Models In order to process and predict the data, in
this section, a decision tree and random forest are applied to the data.

\begin{Shaded}
\begin{Highlighting}[]
\DocumentationTok{\#\#\# Decision tree}
\CommentTok{\#Fit model}
\NormalTok{fit\_DT }\OtherTok{\textless{}{-}} \FunctionTok{rpart}\NormalTok{(}\FunctionTok{as.factor}\NormalTok{(classe) }\SpecialCharTok{\textasciitilde{}}\NormalTok{ ., }\AttributeTok{data=}\NormalTok{STraining, }\AttributeTok{method=}\StringTok{"class"}\NormalTok{)}
\CommentTok{\# Perform prediction(model to predict class)}
\NormalTok{prediction\_DT }\OtherTok{\textless{}{-}} \FunctionTok{predict}\NormalTok{(fit\_DT, STesting, }\AttributeTok{type =} \StringTok{"class"}\NormalTok{)}
\CommentTok{\# Plot result}
\FunctionTok{rpart.plot}\NormalTok{(fit\_DT, }\AttributeTok{main=}\StringTok{"Classification Tree"}\NormalTok{, }\AttributeTok{extra=}\DecValTok{0}\NormalTok{, }\AttributeTok{under=}\ConstantTok{TRUE}\NormalTok{, }\AttributeTok{faclen=}\DecValTok{0}\NormalTok{)}
\end{Highlighting}
\end{Shaded}

\includegraphics{perdict_files/figure-latex/decisiontree-1.pdf}

\begin{Shaded}
\begin{Highlighting}[]
\CommentTok{\#Shows the errors of the prediction algorithm with confusionMatrix in testing .}

\FunctionTok{confusionMatrix}\NormalTok{(prediction\_DT, }\FunctionTok{as.factor}\NormalTok{(STesting}\SpecialCharTok{$}\NormalTok{classe))}
\end{Highlighting}
\end{Shaded}

\begin{verbatim}
## Confusion Matrix and Statistics
## 
##           Reference
## Prediction    A    B    C    D    E
##          A 1232  142    8   39   17
##          B   44  598   74   59   72
##          C   48   93  694  122  111
##          D   44   73   63  525   51
##          E   27   43   16   59  650
## 
## Overall Statistics
##                                          
##                Accuracy : 0.7543         
##                  95% CI : (0.742, 0.7663)
##     No Information Rate : 0.2845         
##     P-Value [Acc > NIR] : < 2.2e-16      
##                                          
##                   Kappa : 0.689          
##                                          
##  Mcnemar's Test P-Value : < 2.2e-16      
## 
## Statistics by Class:
## 
##                      Class: A Class: B Class: C Class: D Class: E
## Sensitivity            0.8832   0.6301   0.8117   0.6530   0.7214
## Specificity            0.9413   0.9370   0.9076   0.9437   0.9638
## Pos Pred Value         0.8567   0.7060   0.6498   0.6944   0.8176
## Neg Pred Value         0.9530   0.9135   0.9580   0.9327   0.9389
## Prevalence             0.2845   0.1935   0.1743   0.1639   0.1837
## Detection Rate         0.2512   0.1219   0.1415   0.1071   0.1325
## Detection Prevalence   0.2932   0.1727   0.2178   0.1542   0.1621
## Balanced Accuracy      0.9122   0.7836   0.8597   0.7983   0.8426
\end{verbatim}

\begin{Shaded}
\begin{Highlighting}[]
\DocumentationTok{\#\#\# Random forest}
\CommentTok{\#Fit model }
\NormalTok{fit\_RF }\OtherTok{\textless{}{-}} \FunctionTok{randomForest}\NormalTok{(}\FunctionTok{as.factor}\NormalTok{(classe) }\SpecialCharTok{\textasciitilde{}}\NormalTok{ ., }\AttributeTok{data=}\NormalTok{STraining, }\AttributeTok{method=}\StringTok{"class"}\NormalTok{)}

\CommentTok{\# Perform prediction(model to predict class)}
\NormalTok{prediction\_RF }\OtherTok{\textless{}{-}} \FunctionTok{predict}\NormalTok{(fit\_RF, STesting, }\AttributeTok{type =} \StringTok{"class"}\NormalTok{)}
\end{Highlighting}
\end{Shaded}

\begin{Shaded}
\begin{Highlighting}[]
\CommentTok{\#Shows the errors of the prediction algorithm with Random Forest in testing .}
\FunctionTok{confusionMatrix}\NormalTok{(}\FunctionTok{as.factor}\NormalTok{(STesting}\SpecialCharTok{$}\NormalTok{classe), prediction\_RF)}
\end{Highlighting}
\end{Shaded}

\begin{verbatim}
## Confusion Matrix and Statistics
## 
##           Reference
## Prediction    A    B    C    D    E
##          A 1395    0    0    0    0
##          B    5  939    5    0    0
##          C    0    7  846    2    0
##          D    0    0   10  793    1
##          E    0    0    0    6  895
## 
## Overall Statistics
##                                           
##                Accuracy : 0.9927          
##                  95% CI : (0.9899, 0.9949)
##     No Information Rate : 0.2855          
##     P-Value [Acc > NIR] : < 2.2e-16       
##                                           
##                   Kappa : 0.9907          
##                                           
##  Mcnemar's Test P-Value : NA              
## 
## Statistics by Class:
## 
##                      Class: A Class: B Class: C Class: D Class: E
## Sensitivity            0.9964   0.9926   0.9826   0.9900   0.9989
## Specificity            1.0000   0.9975   0.9978   0.9973   0.9985
## Pos Pred Value         1.0000   0.9895   0.9895   0.9863   0.9933
## Neg Pred Value         0.9986   0.9982   0.9963   0.9980   0.9998
## Prevalence             0.2855   0.1929   0.1756   0.1633   0.1827
## Detection Rate         0.2845   0.1915   0.1725   0.1617   0.1825
## Detection Prevalence   0.2845   0.1935   0.1743   0.1639   0.1837
## Balanced Accuracy      0.9982   0.9950   0.9902   0.9937   0.9987
\end{verbatim}

\#6-) Conclusion

\hypertarget{result}{%
\subsection{Result}\label{result}}

The confusion matrices show, that the Random Forest algorithm performens
better than decision trees. The accuracy for the Random Forest model was
0.9927 (95\% CI: (0.9899, 0.9949)) compared to 0.7543 (95\% CI: (0.742,
0.7663)) for Decision Tree model. The random Forest model is choosen.

\#submission Finally, in order to complete the project, in this section,
the submitted files of the project are generated using the random forest
algorithm on the test data.

\begin{Shaded}
\begin{Highlighting}[]
\CommentTok{\# Perform prediction}
\NormalTok{predictSubmission }\OtherTok{\textless{}{-}} \FunctionTok{predict}\NormalTok{(fit\_RF, testing3, }\AttributeTok{type=}\StringTok{"class"}\NormalTok{)}
\NormalTok{predictSubmission}
\end{Highlighting}
\end{Shaded}

\begin{verbatim}
##  1  2  3  4  5  6  7  8  9 10 11 12 13 14 15 16 17 18 19 20 
##  B  A  B  A  A  E  D  B  A  A  B  C  B  A  E  E  A  B  B  B 
## Levels: A B C D E
\end{verbatim}

\end{document}
